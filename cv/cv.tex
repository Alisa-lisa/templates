%% start of file `template.tex'.
%% Copyright 2006-2013 Xavier Danaux (xdanaux@gmail.com).
%
% This work may be distributed and/or modified under the
% conditions of the LaTeX Project Public License version 1.3c,
% available at http://www.latex-project.org/lppl/.


\documentclass[11pt,a4paper,sans]{moderncv}        
\moderncvstyle{classic}                            
\moderncvcolor{blue}                                
\usepackage[utf8]{inputenc}
\usepackage[ngerman]{babel}
\usepackage[scale=0.8]{geometry}

\name{Alisa}{Dammer}
\title{Data Scientist, Data Analyst}                                \address{Wördemanns Weg, 4}{22527 Hamburg}{Germany}
\phone[mobile]{+4917680767743}                    \email{alisa.dammer@gmail.com}
\homepage{alisadammer.com}
\photo[64pt][0.4pt]{me.jpg}

\begin{document}
\makecvtitle

\section{Persönliche Daten}
\cvitem{Geburtsort}{Russland, Nowosibirsk}
\cvitem{Familienstand}{ledig}
\cvitem{Bürgerschaft}{Deutsch, Russisch}


\section{Ausbildung}
\cventry{1997--2007}{Schule}{Gymnasium 3}{Nowosibirsk,Russische Föderation}{\textit{}}{Physikalisch-mathematisches Profil}

\cventry{2007--2011}{B.Sc.Wirtschaft}{Staatliche Universität Nowosibirsk}{Russische Föderation}{\textit{}}{Fachbereich Wirtschaft, Wirtschaft}
\cvitem{These}{\emph{Zusammenstellung des Portfolios für einen privaten Investor}}
\cvitem{Gutachter}{Perfiljev A.A.}
\cvitem{Beschreibung}{In dieser Bachelorarbeit habe ich gezeigt, dass die klassischen Portfolio-Modelle auf der russischen Papierbörse nicht effektiv sind. Als Ersatz habe ich ein eigenes Portfolio-Modell für einen privaten Investor vorgeschlagen.}

\cventry{20011--2016}{B.Sc. Wirtschaftsinformatik}{Universität Hamburg}{Hamburg, Deutschland}{\textit{}}{MIN Fakultät, Wirtschaftsinformatik}
\cvitem{These}{\emph{Vergleichen von Python und R auf der einfachen ökonometrischen Aufgaben}}
\cvitem{Gutachter 1}{Prof. Dr. Stefan Voß}
\cvitem{Gutachter 2}{Dr. Stefan Lessmann}
\cvitem{Beschreibung}{Die zwei berühmte analytische Programmiersprachen werden auf der einfache statistische-ökonometrische Aufgabe verglichen. In der These habe ich herkömmlichen Aufgaben von Data Analysts als Evaluationskriterien verwendet. Zusätzlich habe ich zwei ähnliche Programme in R und Python geschrieben, die aus vorbereiteten CSV-Tabellen die lineare Regression bauen. Die Regressionen werden bewertet und darauf eine Vorhersage gebaut. Die Ergebnisse habe ich zusätzlich cross-validiert.}

\section{Arbeitserfahrung}
\cvitem{Eversolve (2015-2016)}{Werkstudent in den Rollen von Software Entwickler, Frontend Entwickler (AngularJS, Bootstrap)}

\section{Kenntnisse und Fähigkeiten}
\cvitem{Programmier-sprachen}{Python, R, JavaScript, Lua, Java}
\cvitem{Sprachen}{Russisch, Englisch, Deutsch}
\cvitem{Markup-Sprachen}{HTML, CSS}
\cvitem{SQL}{MariaDB, PostgresQL}
\cvitem{Zusätzliche Technologien}{git, Trello, ZenHub, Slack, Asana}

\section{Interests}
\cvitem{Sport}{Dancing, Fitness, Snowboard}
\cvitem{Graphics}{Photo manipulation and graphic design}
\cvitem{Über mich}{Stressresistent, freundlich, fähig und willig Neues zu lernen.}

\end{document}